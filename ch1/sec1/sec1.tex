
\documentclass[../ch1.tex, ../../main.tex]{subfiles}

\usepackage{amsthm}

\newcommand{\bbQ}{\mathbb{Q}}
\newcommand{\bbZ}{\mathbb{Z}}

\begin{document}

\subsection{Introduction}

The chapter begins with the discussion of the irrationality of
$\sqrt{2}$, namely that there is no rational solution to the
equation $p^2 = 2$.

\newtheorem{thm}{Theorem}[subsection]
\begin{thm}[The Irrationality of $\sqrt{2}$]
    There is no rational $p$ such that $p^2 = 2$.
\end{thm}
% Same as previously, contents elided for brevity
\begin{proof}
    Suppose, by way of contradiction, there exists $p \in \bbQ$ such that
    \begin{align}
        p^2 = 2 \label{eq:1}
    \end{align}
    Since $p \in \bbQ$, then there exists integers $m, n$ such that
    $$p = \frac{m}{n}$$ where, without loss of generality,
    $m$ and $n$ are not both even.

    From the following and~\eqref{eq:1}, we see that
    \begin{align}
        m^2 = 2n^2.
    \end{align}
    This implies that $m$ is even and further more that $n$ is even.
    This contradictions our assumption that both
    $m$ and $n$ are not both even. \qedhere
\end{proof}
\end{document}